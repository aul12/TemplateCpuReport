%2. Wie bin ich generell da ran gegangen? Was für Konzepte habe ich mir überlegt, welche Strukturen, patterns...
\subsection{Template Specialization}
Metaprogramming in C++ is primarily based on the concept of template-specializiation: for a specific template instantiation
a conrete implementation of the class or function can be given. This makes it possible to implement control structures
using the type system. 

Listing \ref{lst:template_specialization} implements a logical not using structs-template with specialization.
\todo{Disclaimer ! geht auch}

\lstinputlisting[caption={Template specialization}, label=lst:template_specialization]{template_specialization.cpp}

For struct- and class-templates, but not for functions-templates, it is also possible to do partial template 
specialization, if a struct of class depends on more than one type or value a specialization is allowed to define
only a subset of the arguments and leave other arguments as template arguments. Listing
\ref{lst:partial_template_specialization} uses partial template specialization to define a struct template which
behaves similar to \lstinline{std::enable_if}: if the first argument is true the class defines a type, if the
argument is false this type is not defined.

\lstinputlisting[caption={Partial template specialization}, label=lst:partial_template_specialization]
    {partial_template_specialization.cpp}

\subsection{Compile-Time Datatypes}
For implementing algorithms it is not only necessary to be able to implement the program logic and control flow
it is equally important to define data types to work with. When implementing a CPU emulator the required data type
is primarily a sequential container for representing registers, memory and the program itself. 
As the STL does not provide any data types suitable for compile time manipulation a custom container had to be implemented.

The implementation of the sequential container is based on singly linked list. 
This not the obvious choice when implementing sequential containers but array manipulation is not possible during 
compile time, thus a different implementation of a sequential container had to be chosen.

The definition of the list is similar to a typical singly linked list definition in normal run-time C++. Every
element consists of a value and the next element, for the last element the next element is of type \lstinline{ListEnd}.
Listing \ref{lst:typelist} is the definition of the type list, that is a list in which every element is a type, not
a value. This type of list is used for the program of the CPU emulator, as the instructions are types not values.

\lstinputlisting[caption={Type List}, label={lst:typelist}]{type_list.cpp}

Building upon the type list we can define a value list as a type list of \lstinline{ValueContainer}s. Listing
\ref{lst:valuecontainer} gives the definition of the value container struct. This struct contains a value but
is a type itself, thus a type list can be used. This scheme makes it possible to reuse most of the type list
implementation for the value list.

\lstinputlisting[caption={Value Container}, label={lst:valuecontainer}]{value_container.cpp}


\todo[inline]{Helper Functions}

\subsection{Concepts}
\todo[inline]{Motivation}

\todo[inline]{Usage}

\subsection{Limitations}
\todo[inline]{Max Depth}
