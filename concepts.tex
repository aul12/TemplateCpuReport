%2. Wie bin ich generell da ran gegangen? Was für Konzepte habe ich mir überlegt, welche Strukturen, patterns...
\subsection{Template Specialization}
Metaprogramming in C++ is primarily based on the concept of template-specializiation: for a specific template instantiation
a concrete implementation of the class or function can be given. This makes it possible to implement control structures
using the type system. 

Listing \ref{lst:template_specialization} implements a logical not using a struct template with specialization. 
This is just used as an example, the logical not operator "\lstinline{!}" could be used as well instead of defining a struct.

\lstinputlisting[caption={Template specialization}, label=lst:template_specialization]{code/template_specialization.cpp}

For struct- and class-templates, but not for functions-templates, it is also possible to do partial template 
specialization, if a struct of class depends on more than one type or value a specialization is allowed to define
only a subset of the arguments and leave other arguments as template arguments. Listing
\ref{lst:partial_template_specialization} uses partial template specialization to define a struct template which
behaves similar to \lstinline{std::enable_if}: if the first argument is true the class defines a type, if the
argument is false this type is not defined.

\lstinputlisting[caption={Partial template specialization}, label=lst:partial_template_specialization]
    {code/partial_template_specialization.cpp}

\subsection{Compile-Time Data Types}
For implementing algorithms it is not only necessary to be able to implement the program logic and control flow
it is equally important to define data types to work with. When implementing a CPU emulator the required data type
is primarily a sequential container for representing registers, memory and the program itself. 
As the STL does not provide any data types suitable for compile time manipulation a custom container had to be implemented.

The implementation of the sequential container is based on singly linked list. 
This not the obvious choice when implementing sequential containers but array manipulation is not possible during 
compile time, thus a different implementation had to be chosen.

The definition of the list is similar to a typical singly linked list definition in normal run-time C++. Every
element consists of a value and the next element, for the last element the next element is of type \lstinline{ListEnd}.
Listing \ref{lst:typelist} is the definition of an element for a type list, that is a list in which every element is a type, not
a value. This type of list is used for the program of the CPU emulator, as the instructions are distinct types.

\lstinputlisting[caption={Type List}, label={lst:typelist}]{code/type_list.cpp}

Building upon the type list we can define a value list as a type list of \lstinline{ValueContainer}s. Listing
\ref{lst:valuecontainer} gives the definition of the value container struct. This struct contains a value but
is a type itself, thus a type list can be used. This scheme makes it possible to reuse most of the type list
implementation for the value list.

\lstinputlisting[caption={Value Container}, label={lst:valuecontainer}]{code/value_container.cpp}

Additionally a set of structs have been defined to handle compile time lists. These structs behave during compile time like 
functions behave during run time. 

The first struct, and the simplest to understand is the \lstinline{Size} struct, this
struct has a member \lstinline{val} which returns the length of the list, similar to 
\lstinline{std::size} for STL containers \cite{stdsize}. The implementation is based on recursively 
walking the list until next is of type \lstinline{ListEnd}. The implementation is given
in Listing \ref{lst:size}.

\lstinputlisting[caption=Size, label={lst:size}]{code/size.cpp}

The second important struct for handling type lists is the \lstinline{GetType} struct,
this struct can be used to get the type at an index. Based on the \lstinline{GetType}
struct there is also a \lstinline{GetVal} struct for getting the value at an index of
a value list. The \lstinline{GetType} struct defines the member type which contains the type
at the given index. The implementation, given in Listing \ref{lst:get}, 
is recursive, similar to the implementation of \lstinline{Size}.
The recursive implementation fails if the index is larger than the size of the list, 
thus a \lstinline{static_assert} has been added for error checking.


\lstinputlisting[caption={Get Type}, label={lst:get}]{code/get.cpp}

To manipulate the list a \lstinline{SetType} struct is provided to change the type at 
a given index. As types are immutable and can not be changed once created this struct
does not change the original type but creates a new type. The implementation, given
in Listing \ref{lst:set}, is once again recursive: 
the struct recursively walks the list until it reaches the given index. The
type at this position is then replaced by the new type. Similar to above error
checking is performed using a \lstinline{static_assert}.

\lstinputlisting[caption={Set Type}, label={lst:set}]{code/set.cpp}

\subsection{Concepts}
In the struct definitions above many of the template type arguments can not be arbitrary
but are required to be of another template class or fulfill other criteria. In pre C++20
such limitations would be enforced by using type traits and the "substitution failure is 
not an error" (SFINAE) pattern.
Starting with C++20 a new mechanism for defining type traits is added: concepts \cite{concepts}.
While concepts do not provide more features than SFINAE they make the syntax more readable and shorter; 
additionally they enable the compiler to generate better error messages.

The complete implementation of the CPU emulator is based on templates, thus it is especially important to be able to guarantee strong typing, even during the compile time. For this
concepts have been widely used for the implementation of the emulator.

Especially when working with compile time lists the added type safety of concepts is
beneficial. Consider the implemetation of \lstinline{Size} (Listing \ref{lst:size}).
If the template argument is not either a template instantiation of a type list or 
\lstinline{ListEnd} the compiler will throw an error as \lstinline{list::next} is not
defined. From this error it is not directly obvious that the problem is the wrong
argument. To enable the usage of concepts for type lists first the relevant concepts
need to be defined. The most important concept is the \lstinline{TypeList} concept,
it defines whether a type is a valid type list.

The implementation is given in Listing \ref{lst:concept}: first a struct template with
a single member is defined. The member is only true if the argument is a valid type list.
This struct is then used for the definition of the concept.

\lstinputlisting[caption={Implementation of a concept}, label={lst:concept}]{code/concepts.cpp}

The name of the concept can then be used instead of the \lstinline{typename} keyword 
in the template parameter declaration to  signal that the type has to fulfill the concept. 

For the size struct the adapted definition is given in Listing \ref{lst:size_concepts}. 
The compiler is now able to directly print an error if a type does not fulfill the concept.
The other structs can be adapted analogously.
\lstinputlisting[caption={Size using concepts}, label={lst:size_concepts}]{code/size_concepts.cpp}

\subsection{Limitations} \label{sec:limitations}
The C++ standard recommends a maximum supported template instantiation depth of 1024.
Both GCC and Clang implement this recommendation but can be configured 
(using the \texttt{-ftemplate-depth} flag) to use different values. But even with
this flag the practical limitation is at about 10000, with this maximum depth
GCC requires approximatly 40GB of memory when compiling a list of said length, Clang will
crash without a proper error message.

As a result both the memory and program size are limited when using the CPU emulator,
thus certain programs can not be executed.
