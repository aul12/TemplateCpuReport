%3. Dann natürlich die Lösung vorstellen, gerne auch interessante Code-Schnipsel zeigen

\subsection{Execution of Code}
For the emulation the most important data structure is the CPU state. The state consists of the program, the registers,
the memory and the program counter among others. During each CPU state the next state is determined based on the 
current state and the current instruction.
The emulator terminates once the program counter is equal to the length of the program,
the emulator then returns a result containing the program counter, the current content
of the registers and the content of the memory.

The program, the registers and the memory are all defined using  compile time lists.
For the program the list is a type list as the instructions are represented by different
types. The registers are a value list, the value type can be configured (see the
\texttt{config.hpp} header), and is \lstinline{int} by default. The registers are not
accessed directly using indizes, but via an enum class which provides
more readable names for the registers. The memory is a value list with the same type
as the registers to avoid problems related to conversions. The size of the memory
can be configured, depending on the requirements but the limitations of the compiler 
(see Section \ref{sec:limitations}), need to be respected.

For the execution of the instructions a second struct called \lstinline{ExecuteInstr} is defined. 
This struct fetches the instruction based on the program counter and the program. Then
the instruction is executed and the program counter, the registers and the memory is
updated based on the result of the execution.

The logic used for the execution of instructions is show in Listing \ref{lst:cpu}.

\lstinputlisting[caption={Implementation of the main emulator code}, label={lst:cpu}]{code/cpu.cpp}

\subsection{Instructions}
For programming the CPU emulator a total of 28 instructions have been implemented. They consist of arithmetic and logic
operations, load and store operations and jumps, both conditional and unconditional. The instructions are primarily 
influenced by the MIPS architecture, similar to MIPS the instructions are based
on a Load/Store architecture \cite{mips}. For the full list of instructions refer to the \texttt{README.md} file provided with the code. 

For the execution there is a third struct called \lstinline{InstrImpl} which is
specialized for all instructions, depending on the instruction the program counter,
registers and memory is then manipulated in each template specizalization.

Listing \ref{lst:add} shows the implementation of the \lstinline{Add} instruction.
For this the \lstinline{InstrImpl} struct is specialized for the respective type.
This struct then defines three members. The first member are the registers after the
instruction is executed. For the execution first the two operands are fetched using
\lstinline{GetVal}, then the values are summed up and the result is stored into a
register using \lstinline{SetVal}. The member containing the next programm counter
is simply calculated by incrementing the program counter by one. The memory is not
changed by the instruction, thus it is the same as before.

\lstinputlisting[caption={Implementation of the \lstinline{Add} instruction},label={lst:add}]{code/add.cpp}

Listing \ref{lst:jmp} shows the implementation of the \lstinline{Jump} instruction.
Similar to the \lstinline{Add} instruction this is achieved by specializing the 
\lstinline{InstrImpl} struct. In the implementation both the registers and the memory
are not changed. Only the program counter is changed, for this the address is fetched
from the respective register using \lstinline{GetVal} and then the program counter is
set to this value.

\lstinputlisting[caption={Implementation of the \lstinline{Jump} instruction},label={lst:jmp}]{code/jmp.cpp}

Listing \ref{lst:store} shows the implementation of the \lstinline{Store} instruction,
this instruction copies a value from a register into the memory at an address that is
read from a register.
In the implementation the registers are not changed and the program counter is incremented
by one. The memory is changed: first the value to store and the address at which to write
are fetched from the registers. Then the value is written at the address.

\lstinputlisting[caption={Implementation of the \lstinline{Store} instruction}, label={lst:store}]{code/store.cpp}

\subsection{Definining and Running a Program}
The program is a list of instruction types and can be defined using the 
\lstinline{DeclareProgram} type which accepts a variadic list of instructions. 
This type can then be given as a template argument to the \lstinline{CPU} struct, which will start the execution by defining
the initial CPU state. The result of the execution is the type \lstinline{run} defined
by the CPU. It is a struct containing the registers, the memory and the current program
counter. For printing these fields there is the \lstinline{printer} struct which
provides functions for printing the values. These functions are the only code that is
executed during runtime, all other code is executed strictly during compile time.

Given a program \lstinline{my_program} Listing \ref{lst:main} shows how a program can be
executed and the results can be printed. Example programs for calculating fibonacci numbers
and for the implementation of a Turing machine can be found in the file \texttt{programs.hpp} provided with the code.

\lstinputlisting[caption={Code for executing a program and printing the results},label={lst:main}]{code/main.cpp}

\subsection{Debugger}
For more complex programs it is important to be able to debug the program. For this
a debugging mechanism has been implemented in the CPU. When using the struct 
\lstinline{CpuDebug} instead of the normal \lstinline{Cpu} struct a second argument
can be specified. This argument is the number of instructions to run, once the number
of instructions have been executed the execution stops and the current registers and
memory is returned identical to a normal termination. Additionally the \lstinline{is_breakpoint} member of the result is set to \lstinline{true}.

