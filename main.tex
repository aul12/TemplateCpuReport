\NeedsTeXFormat{LaTeX2e}
\documentclass[a4paper,12pt,
headsepline,           % Linie zw. Kopfzeile und Text
oneside,               % einseitig
pointlessnumbers,      % keine Punkte nach den letzten Ziffern in Überschriften
bibtotoc,              % LV im IV
%DIV=15,               % Satzspiegel auf 15er Raster, schmalere Ränder   
BCOR15mm               % Bindekorrektur
%,draft
]{scrbook}
\KOMAoptions{DIV=last} % Neuberechnung Satzspiegel nach Laden von Paket helvet

\pagestyle{headings}
\usepackage{blindtext}

% für Texte in deutscher Sprache
%\usepackage[ngerman]{babel}
\usepackage[utf8]{inputenc}
\usepackage[T1]{fontenc}

% Helvetica als Standard-Dokumentschrift
\usepackage[scaled]{helvet}
\renewcommand{\familydefault}{\sfdefault} 

\usepackage{graphicx}

% Literaturverzeichnis mit BibLKaTeX
%\usepackage[babel,german=quotes]{csquotes}
\usepackage{csquotes}
\usepackage[backend=bibtex8]{biblatex}
\bibliography{bibliography}

% Für Tabellen mit fester Gesamtbreite und variabler Spaltenbreite
\usepackage{tabularx} 

% Besondere Schriftauszeichnungen
\usepackage{url}              % \url{http://...} in Schreibmaschinenschrift
\usepackage{color}            % zum Setzen farbigen Textes

\usepackage{amssymb, amsmath} % Pakete für Mathe-Umgebungen und -Symbole

\usepackage{setspace}         % Paket für div. Abstände, z.B. ZA
%\onehalfspacing              % nur dann, wenn gefordert; ist sehr groß!!
\setlength{\parindent}{0pt}   % kein linker Einzug der ersten Absatzzeile
\setlength{\parskip}{1.4ex plus 0.35ex minus 0.3ex} % Absatzabstand, leicht variabel

% Tiefe, bis zu der Überschriften in das Inhaltsverzeichnis kommen
\setcounter{tocdepth}{3}      % ist Standard

% Beispiele für Quellcode
\usepackage{listings}
\lstset{language=Java,
  showstringspaces=false,
  frame=single,
  numbers=left,
  basicstyle=\ttfamily,
  numberstyle=\tiny}

% hier Namen etc. einsetzen
\newcommand{\fullname}{Paul Nykiel}
\newcommand{\email}{paul.nykiel@uni-ulm.de}
\newcommand{\titel}{Advanced Concepts of Metaprogramming for the Implementation of a CPU Emulator Using Templates}
\newcommand{\jahr}{2020}
\newcommand{\matnr}{941496}
\newcommand{\gutachter}{Dr.\,Borchert}

% hier die Fakultät auswählen
\newcommand{\fakultaet}{Ingenieurwissenschaften, Informatik und\\Psychologie}

% hier das Institut einsetzen
\newcommand{\institut}{--- Institut im Quellcode anpassen nicht vergessen! ---}

% Informationen, die LaTeX in die PDF-Datei schreibt
\pdfinfo{
  /Author (\fullname)
  /Title (\titel)
  /Producer     (pdfeTex 3.14159-1.30.6-2.2)
  /Keywords ()
}

\usepackage{hyperref}
\hypersetup{
pdftitle=\titel,
pdfauthor=\fullname,
pdfsubject={Diplomarbeit},
pdfproducer={pdfeTex 3.14159-1.30.6-2.2},
colorlinks=false,
pdfborder=0 0 0	% keine Box um die Links!
}

% Trennungsregeln
\hyphenation{Sil-ben-trenn-ung}

\begin{document}
\frontmatter

% Titelseite
\thispagestyle{empty}
\begin{addmargin*}[4mm]{-10mm}

\includegraphics[height=1.8cm]{images/unilogo_bild}
\hfill
\includegraphics[height=1.8cm]{images/unilogo_wort}\\[1em]

{\footnotesize
%{\bfseries Universität Ulm} \textbar ~89069 Ulm \textbar ~Germany
\hspace*{115mm}\parbox[t]{35mm}{\bfseries Fakultät für\\
\fakultaet\\
% TODO hier Institut anpassen
\mdseries \institut}\\[2cm]

\parbox{140mm}{\bfseries \LARGE \titel}\\[2.5em]
{\footnotesize Hausarbeit zur Vorlesung "Objektorientierte Programmierung mit C++
"}\\[3em]

{\footnotesize \bfseries Vorgelegt von:}\\
{\footnotesize \fullname\\ \email}\\ \matnr\\[2em]
{\footnotesize \bfseries Gutachter:}\\                     
{\footnotesize \gutachter}\\[2em]
{\footnotesize \jahr}
}
\end{addmargin*}


% Impressum
\clearpage
\thispagestyle{empty}
{ \small
  \flushleft
  Version \today \\\vfill
  \copyright~\jahr~\fullname\\[0.5em]
This work is licensed under the Creative Commons Attribution 4.0 International (CC BY 4.0) License. To view a copy of this license, visit \href{https://creativecommons.org/licenses/by/4.0/}{https://creativecommons.org/licenses/by/4.0/} or send a letter to Creative Commons, 543 Howard Street, 5th Floor, San Francisco, California, 94105, USA. \\
  Satz: PDF-\LaTeXe
}

% ab hier Zeilenabstand etwas größer 
\setstretch{1.2}

\tableofcontents

\mainmatter
%\input{chapters/einleitung}
% hier weitere Kapitel einbinden

\appendix
% hier Anhänge einbinden
%\input{chapters/sources}

\backmatter
\nocite{Knappen2009}
\nocite{Mittelbach2005}
\nocite{Schlosser2014}
\nocite{Sturm2012}
\nocite{Voss2010}

\printbibliography

\clearpage
\thispagestyle{empty}

Name: \fullname \hfill Matrikelnummer: \matnr \vspace{2cm}

\minisec{Erklärung}

Ich erkläre, dass ich die Arbeit selbständig verfasst und keine anderen als die angegebenen Quellen und Hilfsmittel verwendet habe.\vspace{2cm}

Ulm, den \dotfill

\hspace{10cm} {\footnotesize \fullname}
\end{document}
