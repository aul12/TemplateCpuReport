\NeedsTeXFormat{LaTeX2e}
\documentclass[a4paper,11pt,
headsepline,           % Linie zw. Kopfzeile und Text
oneside,               % einseitig
pointlessnumbers,      % keine Punkte nach den letzten Ziffern in Überschriften
bibtotoc,              % LV im IV
DIV=15,               % Satzspiegel auf 15er Raster, schmalere Ränder   
BCOR15mm               % Bindekorrektur
%,draft
]{scrartcl}
\KOMAoptions{DIV=last} % Neuberechnung Satzspiegel nach Laden von Paket helvet

\pagestyle{headings}
\usepackage{blindtext}

\usepackage[utf8]{inputenc}
\usepackage[T1]{fontenc}

\usepackage[scaled]{helvet}
\renewcommand{\familydefault}{\sfdefault} 

\usepackage{graphicx}

\usepackage{csquotes}
\usepackage[backend=bibtex8]{biblatex}
\bibliography{bibliography}

\usepackage{tabularx} 

\usepackage{url}              % \url{http://...} in Schreibmaschinenschrift
\usepackage{color}            % zum Setzen farbigen Textes

\usepackage{amssymb, amsmath} % Pakete für Mathe-Umgebungen und -Symbole

\usepackage{setspace}         % Paket für div. Abstände, z.B. ZA
\setlength{\parindent}{0pt}   % kein linker Einzug der ersten Absatzzeile
\setlength{\parskip}{1.4ex plus 0.35ex minus 0.3ex} % Absatzabstand, leicht variabel

\setcounter{tocdepth}{3}      % ist Standard

\usepackage{listings}
\lstset{language=C++,
        numberstyle=\small,
        basicstyle=\small,
        keywordstyle=\color{blue},
        stringstyle=\color{red},
        commentstyle=\color{magenta},
        morecomment=[l][\color{green}]{\#},
        breaklines=true,
        numbers=left,
        stepnumber=1,
        morekeywords={constexpr, concept},
}


\usepackage{todonotes}

\newcommand{\fullname}{Paul Nykiel}
\newcommand{\email}{paul.nykiel@uni-ulm.de}
\newcommand{\titel}{Advanced Concepts of Metaprogramming for the Implementation of a CPU Emulator Using Templates}
\newcommand{\jahr}{2020}
\newcommand{\matnr}{941496}
\newcommand{\gutachter}{Dr.\,Borchert}

% hier die Fakultät auswählen
\newcommand{\fakultaet}{Mathematik und Wirtschaftswissenschaften}

% hier das Institut einsetzen
\newcommand{\institut}{Institut für Numerische Mathematik}

% Informationen, die LaTeX in die PDF-Datei schreibt
\pdfinfo{
  /Author (\fullname)
  /Title (\titel)
  /Producer     (pdfeTex 3.14159-1.30.6-2.2)
  /Keywords ()
}

\usepackage{hyperref}
\hypersetup{
pdftitle=\titel,
pdfauthor=\fullname,
pdfsubject={Diplomarbeit},
pdfproducer={pdfeTex 3.14159-1.30.6-2.2},
colorlinks=false,
pdfborder=0 0 0	% keine Box um die Links!
}

% Trennungsregeln
\hyphenation{Sil-ben-trenn-ung}

\begin{document}

% Titelseite
\thispagestyle{empty}
\begin{addmargin*}[4mm]{-10mm}

\includegraphics[height=1.8cm]{images/unilogo_bild}
\hfill
\includegraphics[height=1.8cm]{images/unilogo_wort}\\[1em]

{\footnotesize
%{\bfseries Universität Ulm} \textbar ~89069 Ulm \textbar ~Germany
\hspace*{115mm}\parbox[t]{35mm}{\bfseries Fakultät für\\
\fakultaet\\
% TODO hier Institut anpassen
\mdseries \institut}\\[2cm]

\parbox{140mm}{\bfseries \LARGE \titel}\\[2.5em]
{\footnotesize Hausarbeit zur Vorlesung "Objektorientierte Programmierung mit C++
"}\\[3em]

{\footnotesize \bfseries Vorgelegt von:}\\
{\footnotesize \fullname\\ \email}\\ \matnr\\[2em]
{\footnotesize \bfseries Gutachter:}\\                     
{\footnotesize \gutachter}\\[2em]
{\footnotesize \jahr}
}
\end{addmargin*}


% Impressum
\clearpage
\thispagestyle{empty}
{ \small
  \flushleft
  Version \today \\\vfill
  \copyright~\jahr~\fullname\\[0.5em]

This report is free software: you can redistribute it and/or modify
it under the terms of the GNU General Public License as published by
the Free Software Foundation, either version 3 of the License, or
(at your option) any later version.

This report is distributed in the hope that it will be useful,
but WITHOUT ANY WARRANTY; without even the implied warranty of
MERCHANTABILITY or FITNESS FOR A PARTICULAR PURPOSE.  See the
GNU General Public License for more details.

You should have received a copy of the GNU General Public License
along with this program.  If not, see \href{https://www.gnu.org/licenses}{https://www.gnu.org/licenses}.\\

  Satz: PDF-\LaTeXe
}

\newpage

% ab hier Zeilenabstand etwas größer 
\setstretch{1.2}

\tableofcontents

\newpage

\section{Problem Statement}
%1. Was ist das Problem / die Aufgabe

%Hierbei würde
%ich auf ein Projekt aufbauen, an dem ich schon die letzten Wochen
%arbeite. Es geht um die Implementierung eines einfachen
%CPU-Emulators (die Architektur ist teilweise von MIPS inspiriert)
%rein über das C++-Typsystem (zur verbesserten Typsicherheit werden
%außerdem Concepts genutzt). Programme sind jeweils ein eigener Typ
%und werden vollständig zur Compile-Time ausgeführt, dadurch kann gut
%gezeigt werden dass bereits das C++-Typsystem Turing-Vollständig
%ist. Den aktuellen Stand der Entwicklung finden sie unter:
%https://github.com/aul12/TemplateCpu. Im Rahmen der Arbeit würde ich
%zuerst die genutzten Konzepte vorstellen, und dann erklären wie ein
%Programm ausgeführt wird, für das Softwarepaket würde ich das
%Programm um eine ordentliche Dokumentation erweitern und eine Turing
%Maschine als Program für den CPU-Emulator implementieren.

In comparison to most other widely adopted languages the C++ language provides a very expressive type system \cite{concepts05}. Custom
types (\lstinline{structs} and \lstinline{classes}) do not only consist of other types but can also depend on other
types and values. This yields types which can fundamentaly change their behaviour based on other types \cite[Chapter~13.3]{std}. As a result
it is possible to implement algorithms which only consist of the type system and dependent types, this concept is known
as metaprogramming. In the case of C++ this metaprogramming is turing complete, this means that all algorithms than
can be computed can also be computed by the C++ type system \cite{TuringComputability}.

In this report the implementation of a CPU emulator using only the C++ type system is presented. This emulator
accepts programs, which are custom types that behave like instructions, and executes these during the compilation of
the program. These programs can be, in theory, arbitrary and thus all computable algorithms can be implemented. As a
demonstration of the Turing completeness a Turing machine will be implemented, the Turing machine is a concept which
is used for the definition of Turing computability \cite{Turing1936}.

This report is structured into three parts: first the necessary concepts and data types for implementing more complex 
programs using metaprogramming are introduced. In the second part the implementation of the CPU emulator is described
and an overview on how to implement programs for the emulator is given. In the last part a conclusion is formulated and
possible improvements are listed.


\section{Foundamentals}
%2. Wie bin ich generell da ran gegangen? Was für Konzepte habe ich mir überlegt, welche Strukturen, patterns...
\subsection{Template Specialization}
Metaprogramming in C++ is primarily based on the concept of template-specializiation: for a specific template instantiation
a concrete implementation of the class or function can be given. This makes it possible to implement control structures
using the type system. 

Listing \ref{lst:template_specialization} implements a logical not using a struct template with specialization. 
This is just used as an example, the logical not operator "\lstinline{!}" could be used as well instead of defining a struct.

\lstinputlisting[caption={Template specialization}, label=lst:template_specialization]{code/template_specialization.cpp}

For struct- and class-templates, but not for functions-templates, it is also possible to do partial template 
specialization, if a struct of class depends on more than one type or value a specialization is allowed to define
only a subset of the arguments and leave other arguments as template arguments. Listing
\ref{lst:partial_template_specialization} uses partial template specialization to define a struct template which
behaves similar to \lstinline{std::enable_if}: if the first argument is true the class defines a type, if the
argument is false this type is not defined.

\lstinputlisting[caption={Partial template specialization}, label=lst:partial_template_specialization]
    {code/partial_template_specialization.cpp}

\subsection{Compile-Time Data Types}
For implementing algorithms it is not only necessary to be able to implement the program logic and control flow
it is equally important to define data types to work with. When implementing a CPU emulator the required data type
is primarily a sequential container for representing registers, memory and the program itself. 
As the STL does not provide any data types suitable for compile time manipulation a custom container had to be implemented.

The implementation of the sequential container is based on singly linked list. 
This not the obvious choice when implementing sequential containers but array manipulation is not possible during 
compile time, thus a different implementation had to be chosen.

The definition of the list is similar to a typical singly linked list definition in normal run-time C++. Every
element consists of a value and the next element, for the last element the next element is of type \lstinline{ListEnd}.
Listing \ref{lst:typelist} is the definition of an element for a type list, that is a list in which every element is a type, not
a value. This type of list is used for the program of the CPU emulator, as the instructions are distinct types.

\lstinputlisting[caption={Type List}, label={lst:typelist}]{code/type_list.cpp}

Building upon the type list we can define a value list as a type list of \lstinline{ValueContainer}s. Listing
\ref{lst:valuecontainer} gives the definition of the value container struct. This struct contains a value but
is a type itself, thus a type list can be used. This scheme makes it possible to reuse most of the type list
implementation for the value list.

\lstinputlisting[caption={Value Container}, label={lst:valuecontainer}]{code/value_container.cpp}

Additionally a set of structs have been defined to handle compile time lists. These structs behave during compile time like 
functions behave during run time. 

The first struct, and the simplest to understand is the \lstinline{Size} struct, this
struct has a member \lstinline{val} which returns the length of the list, similar to 
\lstinline{std::size} for STL containers \cite{stdsize}. The implementation is based on recursively 
walking the list until next is of type \lstinline{ListEnd}. The implementation is given
in Listing \ref{lst:size}.

\lstinputlisting[caption=Size, label={lst:size}]{code/size.cpp}

The second important struct for handling type lists is the \lstinline{GetType} struct,
this struct can be used to get the type at an index. Based on the \lstinline{GetType}
struct there is also a \lstinline{GetVal} struct for getting the value at an index of
a value list. The \lstinline{GetType} struct defines the member type which contains the type
at the given index. The implementation, given in Listing \ref{lst:get}, 
is recursive, similar to the implementation of \lstinline{Size}.
The recursive implementation fails if the index is larger than the size of the list, 
thus a \lstinline{static_assert} has been added for error checking.


\lstinputlisting[caption={Get Type}, label={lst:get}]{code/get.cpp}

To manipulate the list a \lstinline{SetType} struct is provided to change the type at 
a given index. As types are immutable and can not be changed once created this struct
does not change the original type but creates a new type. The implementation, given
in Listing \ref{lst:set}, is once again recursive: 
the struct recursively walks the list until it reaches the given index. The
type at this position is then replaced by the new type. Similar to above error
checking is performed using a \lstinline{static_assert}.

\lstinputlisting[caption={Set Type}, label={lst:set}]{code/set.cpp}

\subsection{Concepts}
In the struct definitions above many of the template type arguments can not be arbitrary
but are required to be of another template class or fulfill other criteria. In pre C++20
such limitations would be enforced by using type traits and the "substitution failure is 
not an error" (SFINAE) pattern.
Starting with C++20 a new mechanism for defining type traits is added: concepts \cite{concepts}.
While concepts do not provide more features than SFINAE they make the syntax more readable and shorter; 
additionally they enable the compiler to generate better error messages.

The complete implementation of the CPU emulator is based on templates, thus it is especially important to be able to guarantee strong typing, even during the compile time. For this
concepts have been widely used for the implementation of the emulator.

Especially when working with compile time lists the added type safety of concepts is
beneficial. Consider the implemetation of \lstinline{Size} (Listing \ref{lst:size}).
If the template argument is not either a template instantiation of a type list or 
\lstinline{ListEnd} the compiler will throw an error as \lstinline{list::next} is not
defined. From this error it is not directly obvious that the problem is the wrong
argument. To enable the usage of concepts for type lists first the relevant concepts
need to be defined. The most important concept is the \lstinline{TypeList} concept,
it defines whether a type is a valid type list.

The implementation is given in Listing \ref{lst:concept}: first a struct template with
a single member is defined. The member is only true if the argument is a valid type list.
This struct is then used for the definition of the concept.

\lstinputlisting[caption={Implementation of a concept}, label={lst:concept}]{code/concepts.cpp}

The name of the concept can then be used instead of the \lstinline{typename} keyword 
in the template parameter declaration to  signal that the type has to fulfill the concept. 

For the size struct the adapted definition is given in Listing \ref{lst:size_concepts}. 
The compiler is now able to directly print an error if a type does not fulfill the concept.
The other structs can be adapted analogously.
\lstinputlisting[caption={Size using concepts}, label={lst:size_concepts}]{code/size_concepts.cpp}

\subsection{Limitations} \label{sec:limitations}
The C++ standard recommends a maximum supported template instantiation depth of 1024.
Both GCC and Clang implement this recommendation but can be configured 
(using the \texttt{-ftemplate-depth} flag) to use different values. But even with
this flag the practical limitation is at about 10000, with this maximum depth
GCC requires approximatly 40GB of memory when compiling a list of said length, Clang will
crash without a proper error message.

As a result both the memory and program size are limited when using the CPU emulator,
thus certain programs can not be executed.


\section{Implementation}
%3. Dann natürlich die Lösung vorstellen, gerne auch interessante Code-Schnipsel zeigen

\subsection{CPU State}
For the emulation the most important data structure is the CPU state. The state consists of the program, the registers,
the memory and the program counter among others. During each CPU state the next state is determined based on the 
current state and the current instruction.

The program, the registers and the memory are all defined using  compile time lists.
For the program the list is a type list as the instructions are represented by different
types. The registers is a value list, the value type can be configured (see the
\texttt{config.hpp} header), and is \lstinline{int} by default. The registers are not
accessed directly using indizes in the list, but via an enum class which provides
more readable names for the registers. The memory is a value list with the same type
as the registers to avoid problems related to conversions. The size of the memory
can be configured, depending on the needs but the limitations of the compiler 
(see Section \ref{sec:limitations}), need to be respected.


For the execution of the instructions a second struct called \lstinline{ExecuteInstr} is defined. 
This struct fetches the instruction based on the program counter and the program. Then
the instruction is executed and the program counter, the registers and the memory is
updated based on the result of the execution.

For the execution there is a third struct called \lstinline{InstrImpl} which is
specialized for all instructions, depending on the instruction the program counter,
registers and memory is then manipulated in each template specizalization.

\todo[inline]{Register, Memory, PC}

\todo[inline]{Instructions}


\todo[inline]{Printing results}

\todo[inline]{Debugger}


\section{Conclusion}
%4. Ausblick, was könnte man erweitern (warum ist das idealerweise einfach? => Weil dein code halt so krass gut aufgebaut ist...), oder auch anders machen?

\todo[inline]{Fazit: Applications}

This report explained how a CPU emulator can be implemented using metaprogramming techniques.
The necessary basics, such as compile time data types and concepts, have been explained 
in the first section. The second section explained the structure of the emulator and how
programs are executed by the emulator. Additionally a program for emulating a turing
machine has been implemented for the CPU emulator to prove that the emulator, and thus
C++ metaprogramming is turing complete. It is important to consider that while in theory
arbitrary programs can be executed in reality the programs are limited by the compiler.

\subsection{Future Improvements}
If necessary more instructions can be added easily in the future. A new instructions
requires the type declaration and the specialization of the \lstinline{InstrImpl}
struct with the implementation of the instruction. The current instruction set
is sufficient to implement all programs, but new instructions might reduce the
complexity of programs significantly.

Another improvement which would improve the usability of the emulator is the implementation
of input and output operations. Currently there is no easy way of doing this, there are
similar programs which implement I/O operations by recompiling the program for
every input \todo{src: https://jguegant.github.io/blogs/tech/meta-crush-saga.html,
https://github.com/Drahflow/compile-time-adventures} which could be implemented for
emulator but would require additional runtime code for recompiling the program.

Another usability improvement for the emulator, especially when writing large programs,
would be the addition of labels, like they are used in most assembly dialects.
Such as labels will be implemented in the future as part of an assembler, but are
currently not supported as they are not a feature provided by a CPU but by an assembler.


\newpage
\appendix

\printbibliography

%\clearpage
%\thispagestyle{empty}

%Name: \fullname \hfill Matrikelnummer: \matnr \vspace{2cm}

%\minisec{Erklärung}

%Ich erkläre, dass ich die Arbeit selbständig verfasst und keine anderen als die angegebenen Quellen und Hilfsmittel verwendet habe.\vspace{2cm}

%Ulm, den \dotfill

%\hspace{10cm} {\footnotesize \fullname}

\end{document}
