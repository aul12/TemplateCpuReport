%1. Was ist das Problem / die Aufgabe

%Hierbei würde
%ich auf ein Projekt aufbauen, an dem ich schon die letzten Wochen
%arbeite. Es geht um die Implementierung eines einfachen
%CPU-Emulators (die Architektur ist teilweise von MIPS inspiriert)
%rein über das C++-Typsystem (zur verbesserten Typsicherheit werden
%außerdem Concepts genutzt). Programme sind jeweils ein eigener Typ
%und werden vollständig zur Compile-Time ausgeführt, dadurch kann gut
%gezeigt werden dass bereits das C++-Typsystem Turing-Vollständig
%ist. Den aktuellen Stand der Entwicklung finden sie unter:
%https://github.com/aul12/TemplateCpu. Im Rahmen der Arbeit würde ich
%zuerst die genutzten Konzepte vorstellen, und dann erklären wie ein
%Programm ausgeführt wird, für das Softwarepaket würde ich das
%Programm um eine ordentliche Dokumentation erweitern und eine Turing
%Maschine als Program für den CPU-Emulator implementieren.

In comparison to most other widely adopted languages the C++ language provides a very expressive type system \cite{concepts05}. Custom
types (\lstinline{structs} and \lstinline{classes}) do not only consist of other types but can also depend on other
types and values. This yields types which can fundamentaly change their behaviour based on other types \cite[Chapter~13.3]{std}. As a result
it is possible to implement algorithms which only consist of the type system and dependent types, this concept is known
as metaprogramming. In the case of C++ this metaprogramming is turing complete, this means that all algorithms than
can be computed can also be computed by the C++ type system \cite{TuringComputability}.

In this report the implementation of a CPU emulator using only the C++ type system is presented. This emulator
accepts programs, which are custom types that behave like instructions, and executes these during the compilation of
the program. These programs can be, in theory, arbitrary and thus all computable algorithms can be implemented. As a
demonstration of the Turing completeness a Turing machine will be implemented, the Turing machine is a concept which
is used for the definition of Turing computability \cite{Turing1936}.

This report is structured into three parts: first the necessary concepts and data types for implementing more complex 
programs using metaprogramming are introduced. In the second part the implementation of the CPU emulator is described
and an overview on how to implement programs for the emulator is given. In the last part a conclusion is formulated and
possible improvements in the future are listed.
